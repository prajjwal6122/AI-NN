\documentclass[12pt]{article}
\usepackage[english]{babel}
\usepackage{natbib}
\usepackage{url}
\usepackage[utf8x]{inputenc}
\usepackage{amsmath}
\usepackage{graphicx}
\graphicspath{{images/}}
\usepackage{parskip}
\usepackage{fancyhdr}
\usepackage{vmargin}
\setmarginsrb{3 cm}{2.5 cm}{3 cm}{2.5 cm}{1 cm}{1.5 cm}{1 cm}{1.5 cm}
\usepackage{hyperref}
\usepackage{color}
\title{A.I. \&  N.N}								% Title
\author{19111040 Prajjwal sharma 5th SEM}								% Author
\date{\today}											% Date

\makeatletter
\let\thetitle\@title
\let\theauthor\@author
\let\thedate\@date
\makeatother

\pagestyle{fancy}
\fancyhf{}
\rhead{\theauthor}
\lhead{\thetitle}
\cfoot{\thepage}

\begin{document}

%%%%%%%%%%%%%%%%%%%%%%%%%%%%%%%%%%%%%%%%%%%%%%%%%%%%%%%%%%%%%%%%%%%%%%%%%%%%%%%%%%%%%%%%%

\begin{titlepage}
	\centering
    \vspace*{0.5 cm}
    \includegraphics[scale = 0.5]{logo.png}\\[1.0 cm]	% University Logo
    \textsc{\Large \newline NATIONAL INSTITUTE OF TECHNOLOGY RAIPUR}\\[2.0 cm]	% University Name
	\textsc{\Large BIOMEDICAL ENGINEERING}\\[0.5 cm]				% Course Code
	\rule{\linewidth}{0.2 mm} \\[0.4 cm]
	{ \huge \bfseries \thetitle}\\
	\rule{\linewidth}{0.2 mm} \\[1.5 cm]
	
	\begin{minipage}{0.5\textwidth}
		\begin{flushleft} \large
			\emph{Name:}\\
			Prajjwal Sharma\\
            5th SEMESTER\\
            19111040\\
			\end{flushleft}
			\end{minipage}~
			\begin{minipage}{0.4\textwidth}
            
			\begin{flushright} \large
			\emph{Assignment 1:} \\
				\bf{10 Keywords on Philosophy of AI}
		\end{flushright}
        
	\end{minipage}\\[2 cm]
	
	

    
    
    
	
\end{titlepage}

%%%%%%%%%%%%%%%%%%%%%%%%%%%%%%%%%%%%%%%%%%%%%%%%%%%%%%%%%%%%%%%%%%%%%%%%%%%%%%%%%%%%%%%%%




%%%%%%%%%%%%%%%%%%%%%%%%%%%%%%%%%%%%%%%%%%%%%%%%%%%%%%%%%%%%%%%%%%%%%%%%%%%%%%%%%%%%%%%%%

\section{Philosophy of AI}
\small The \textbf{philosophy} of artificial intelligence is a branch of the philosophy of technology that explores artificial intelligence and its implications for knowledge and understanding of intelligence, ethics, consciousness, epistemology, and free will. Furthermore, the technology is concerned with the creation of artificial animals or artificial people (or, at least, artificial creatures; see artificial life) so the discipline is of considerable interest to philosophers.These factors contributed to the emergence of the philosophy of artificial intelligence. Some scholars argue that the AI community's dismissal of philosophy is \textbf{detrimental}.
\section{Different Propositions on AI}
\small Important \textbf{propositions} in the philosophy of AI include some of the following
\begin{description}
\item[$\bullet$] Turing's "polite convention": If a machine behaves as intelligently as a human being, then it is as intelligent as a human being
\item[$\cdot$] The Dartmouth proposal: "Every aspect of learning or any other feature of intelligence can be so precisely described that a machine can be made to simulate it."
\item[$\bullet$] Newell and Herbert A. Simon's physical symbol system hypothesis: "A physical symbol system has the necessary and sufficient means of general intelligent action."
\item[$\bullet$] John Searle's strong AI hypothesis: "The appropriately programmed computer with the right inputs and outputs would thereby have a mind in exactly the same sense human beings have minds."
\item[$\bullet$] Searle's strong AI hypothesis: "The appropriately programmed computer with the right inputs and outputs would thereby have a mind in exactly the same sense human beings have minds.
\end{description}
\section{Questions about AI}
\small\subsection{Can a machine display general intelligence?}
Is it possible to create a machine that can solve all the problems humans solve using their intelligence? This question defines the scope of what machines could do in the future and guides the direction of AI research. It only concerns the behavior of machines and ignores the issues of interest to psychologists, cognitive scientists and philosophers; to answer this question, it does not matter whether a machine is really thinking (as a person thinks) or is just acting like it is thinking.\\
\textbf{What is Intelligence}\\
Twenty-first century AI research defines intelligence in terms of intelligent agents. An "agent" is something which perceives and acts in an environment. A "performance measure" defines what counts as success for the agent.\\
some more measures for intelligence:-\\
\textcolor{blue}{\href{https://en.wikipedia.org/wiki/Turing_test}{Turing test}}\\
\subsection{Can a machine have a mind, consciousness, and mental states?}
 Some of the harshest critics of artificial intelligence agree that the brain is just a machine, and that consciousness and intelligence are the result of physical processes in the brain. The difficult philosophical question is this: can a computer program, running on a digital machine that shuffles the binary digits of zero and one, duplicate the ability of the neurons to create minds, with mental states (like understanding or perceiving), and ultimately, the experience of consciousness? but the debate on AI continues
\section{Is thinking a kind of computation?}
The computational theory of mind or \textbf{"computationalism"} claims that the relationship between mind and brain is similar (if not identical) to the relationship between a running program and a computer. The idea has philosophical roots in Hobbes (who claimed reasoning was "nothing more than reckoning"), Leibniz (who attempted to create a logical calculus of all human ideas), Hume (who thought perception could be reduced to "atomic impressions") and even Kant (who analyzed all experience as controlled by formal rules).[66] The latest version is associated with philosophers Hilary Putnam and Jerry Fodor.This question bears on our earlier questions: if the human brain is a kind of computer then computers can be both intelligent and conscious, answering both the practical and philosophical questions of AI. In terms of the practical question of AI ("Can a machine display general intelligence?"), some versions of computationalism make the claim that (as Hobbes wrote):
\section{summary}
\small Can a machine have emotions?
Can a machine be self-aware?
Can a machine be original or creative?\\
these are some more advanced question which will clear our deep understanding of AI.Some scholars argue that the AI community's dismissal of philosophy is detrimental.through this article we grasp a knowledge of the philosophy of \textbf{ARTIFICIAL  INTELLIGENCE}

\end{document}